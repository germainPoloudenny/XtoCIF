\begin{abstract}
Automated crystal structure prediction (CSP) can accelerate materials discovery by combining experimental data with generative models. \xtoCIF builds on \deCIFer and retains the original autoregressive architecture and tokenizer. We introduce engineering improvements and a standardized powder X‑ray diffraction (PXRD) evaluation protocol. The main changes are a GPU‑optimized training pipeline with approximately \(44\times\) higher throughput, deterministic beam search decoding, diffraction‑aware reranking using the weighted‑profile residual \(\Rwp\), and a unified evaluation suite that covers descriptor‑free, composition, and composition‑plus‑space‑group conditioning.

On NOMA, the descriptor‑free setting benefits most: \(\Rwp\) decreases from 0.23 to 0.18 while the structural match rate (MR) increases from 5.6~\% to 8.1~\%. With composition or composition‑plus‑space‑group conditioning, MR rises from the \deCIFer baseline of \(\approx94~\%\) to \(\approx97~\%\) while \(\Rwp\) drops further. These gains arise from training, decoding, and evaluation engineering rather than architectural changes—the generator and tokenizer are unchanged. In practice we recommend a simple default: deterministic beam search with width \(B{=}5\) followed by \(\Rwp\)‑based reranking, which consistently improves over greedy sampling at modest compute cost.
\end{abstract}
