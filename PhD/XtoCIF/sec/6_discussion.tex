\section{Discussion and Outlook}

\paragraph{Bridging simulation and experiment.}
\deCIFer first demonstrated that PXRD-conditioned autoregressive decoding over CIF tokens can close the loop between diffraction traces and atomistic structures. \xtoCIF inherits this interface but strengthens the practical story: faster training makes descriptor‑free checkpoints easier to refresh, and the deterministic beam plus \(\Rwp\) reranker reduces failure cases when no chemical metadata are available. The reduction in \(\Rwp\) for descriptor‑free inference suggests that language models can internalize peak-to-structure mappings traditionally handled by iterative refinement. Closing the remaining gap to experimental corpora such as CHILI-100K will require measured instrument response functions or differentiable forward models that adapt to per-scan idiosyncrasies, an avenue we leave for future work.

\paragraph{Scaling inference responsibly.}
Test-time scaling via beam search and \(\Rwp\) reranking delivers clear benefits, but it also increases compute and memory consumption linearly with beam width. In laboratory settings where dozens of unknowns must be solved per day, we must balance accuracy with cost. Adaptive beam allocation, where we grow the beam only when the PXRD evidence is ambiguous, or early-exit criteria based on intermediate \(\Rwp\) signals could keep inference budgets manageable without sacrificing fidelity. Because the descriptor-free regime saw the largest relative gain (Table~\ref{tab:decoding_refinements}), we view it as the most impactful setting for future research on adaptive beams and cost-aware reranking.

\paragraph{Generalization limits and future data.}
Evaluating on experimental PXRD remains an open milestone. Low-symmetry, multi-element oxides with broad peaks dominate emerging energy materials, so future work should curate additional experimental corpora (e.g., CHILI-100K) with richer metadata such as instrument geometry, sample preparation notes, and phase-fraction estimates. Mixing such corpora into training, possibly via curriculum schedules that anneal from clean to noisy traces, could tighten the remaining gap between simulated and measured PXRD. Given that descriptor-free PXRD is ubiquitous yet difficult, we recommend prioritizing data collection and modeling advances that operate without chemical hints before chasing marginal gains in already-saturated conditioned settings.

\paragraph{User-facing workflows.}
Ultimately the value of \xtoCIF depends on seamless integration into refinement suites. Our current prototype emits a ranked list of CIFs; crystallographers still need to inspect and, if necessary, refine these candidates manually. Integrating with existing Rietveld software via standardized APIs or generating scripts that reproduce the top candidates in GSAS-II/FullProf would accelerate adoption. Likewise, exposing uncertainty estimates (e.g., via ensemble disagreement across beams) would help practitioners judge when automated suggestions are trustworthy.

\paragraph{Broader implications.}
Beyond PXRD, the conditioning pipeline is agnostic to the specific diffraction modality; electron and neutron scattering profiles could be embedded with minimal changes, enabling multi-modal structure inference. Furthermore, the autoregressive framework lends itself to active-learning loops in autonomous labs: by simulating PXRD under hypothetical synthesis routes and ranking the most informative measurements, the same model could inform experiment scheduling. We expect these directions to shape the next milestones on the path toward reliable, fully automated structure solution for complex materials workflows.
